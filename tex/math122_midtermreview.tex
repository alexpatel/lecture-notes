\documentclass[12pt]{article}
\usepackage{graphicx} 
\usepackage{latexsym}
\usepackage{amssymb,amsmath}
\usepackage{hyperref}
\usepackage{mathtools}
\usepackage[margin=0.75in]{geometry}
\allowdisplaybreaks

\usepackage[T1]{fontenc}
\usepackage{ccfonts}

\begin{document}

\begin{center}
\textbf{Math 122: Midterm Review} \\
Alexander H. Patel \\
{\tt alexanderpatel@college.harvard.edu} \\
\today
\end{center}

\begin{enumerate}
    \item
        What are all the class equations for the symmetric groups?

        \begin{align*}
            |D_n| &= 2n = 1 + \times  \text{if n is odd, } \text{ otherwise. } \\
            |A_4| \cong |T| &= 12 = 1 + 3 + 4 + 4 \\
            |S_4| \cong |O| &= 24 = 1 + 3 + 6 + 6 + 8 \\
            |A_5| \cong |I| &= 60 = 1 + 12 + 12 + 15 + 20 \\
            |S_5| &= 120 = 1 + 10 + 15 + 20 + 20 + 24 + 30
        \end{align*}

    \item
        Find a composition series of $S_3$. The kernel of the sign
        homomorphism $S_3 \rightarrow C_2$ is normal - it is $A_3$. $S_2/A_3 =
        C_2$. $A_3 = \{1, (123), (132)\}$, so $A_3$ is just $C_3$.

        If you don't have a homomorphism, then look at the class equation. $S_3
        = 1 + 2 + 3$. 1 is the identity, 2 is the three cycles, and 3 is the
        transpositions.

        Normal subgroups contain the identity and the size divides six. So the
        transpositions and the identity cannot form a normal subgroup because 1
        + 4 doesn't divide 6. If their sum divides 6, then it is normal, but
        you still have to check that it is a subgroup (check closed).
    \item
        Non-abelian group, $|G| =28$, all Sylow 2-subgroups are cyclic. Prove
        this group is unique.

        $28 - 2^2 \times 7$, the number of 2-subgroups is 1 or 7 and 7-groups
        is 1. So the 7-subgroup is $C_7$, normal in $G$. 

        If there is 1 Sylow 2-subgroups (is also normal), then the group is
        abelian. So there are 7. Choose a generator $C_7 = \langle a \rangle$
        and $C_4 = \langle b \rangle$, What is $bab^{-1}$? $C_7$ is normal, so
        $bab^{-1} = a^k$ for $k \in \{1, 7\}$.

        There are 28 elements in form $a^ib^j$, but then it would abelian. So $bab^{-1} \ne a$. 

        If you have $bab^{-1}a^k$, then $b^2ab^{-1}$ = $b(bab^{-1})b^{-1} =
        ba^kb^{-1} = (bab^{-1})^k = a^{k^2}$. You continue this process and get
        $b^4ab^{-4} = a^{k^4}$. $b$ has order 4, so $k^4 = 1 \mod 7$. The check
        all 0, 1, 2, 3, 4, 5, 6 by bringing to power of 4 and check mod 7.

        The only power it works for is 6, so $bab^{-1} = a^6 = a^{-1}
        \rightarrow ba = a^{-1}b$, so this is relation of dihedral group. 
    \item
        $X$ is a $G$-set. Can decompose the elements of $X$ into orbits. For
        element $x$, there's the size of the orbit containing $x$ and the the
        size of the stabilizer of $x$.

        Fixed point theorem
\end{enumerate}

\end{document}
