\documentclass[12pt]{article}
\usepackage{latexsym}
\usepackage{amssymb,amsmath}
\usepackage{mathtools}
\usepackage{amsthm}
\usepackage[margin=1.0in]{geometry}
\allowdisplaybreaks

\theoremstyle{definition}
\newtheorem{theorem}{Theorem}
\newtheorem{defn}{Definition}
\newtheorem{prop}{Proposition}
\newtheorem{cor}{Corollary}
\newtheorem{lemma}[theorem]{Lemma}

\usepackage[T1]{fontenc}
\usepackage{ccfonts}
\setlength\parindent{0pt}

\renewcommand{\labelenumi}{\thesection.\arabic{enumi}}
\renewcommand{\labelenumii}{\thesection.\arabic{enumi}}
\renewcommand{\labelenumiii}{\thesection.\arabic{enumi}}

\newcommand{\w}{Wittgenstein}
\newcommand{\T}{Tractatus}
\newcommand{\g}{Goldfarb}

\begin{document}

\begin{center}
\textbf{PHIL 241: Wittgenstein's Tractatus} \\
Alexander H. Patel \\
{\tt alexanderpatel@college.harvard.edu} \\
Last Updated: \today
\end{center}

These are lecture notes for the Fall 2017 offering of Harvard Philosophy 241:
``Wittgenstein's Tractatus: Seminar" with Professor Warren Goldfarb. Pardon any
mistakes or typos.

\tableofcontents

\section{September 6, 2017}

\begin{itemize}
    \itemsep0em 
    \item This is a semina on early Wittgenstein. We'll talk about the basic outline of what the Tractatus seems to say and try and interpret what it means.
    \item This is a graduate seminar. There is no syllabus. What we do is what you urge me to do after we get started. Then you're allowed to push me in one direction or another.
    \item For the graduate students the requirement is a big paper in the end, but for the undergraduates he has to think about it. Once he gets teh enrollment figures he will talk to the undergraduates about what the requirements are.
    \item If you don't already own the Tractatus then you do not belong in this seminar. What he did order at the Coop is Wittgenstein's notebooks, but it is very hard to know how to take them. we'll also read the letter to Russell, which are somwhat marginal but nonetheless important. We're not going to go much later than that. We might get to Wittgenstein on Logical Forms, where he ultimately seems to give up.
    \item If you go to one decimal point in the Tractatus, then it does seem to take a kind of shape. 
    \item Here's a story: how did Goldfarb get into this book? His sophomore tutor was giving a tutorial on Tractatus, and Goldfarb got assigned to the tutorial because he was interested in logic. But nobody in the seminar (or the tutor) knew anything about Russell or Frege. But he was fascinated because it looks like arguments, it looks like logic. How much is it actually logic? We'll see, this is debatable. 
    \item The name \textit{Tractatus} was given by G.E. Moore after Spinoza, who was also fixed on showing things deductively.
    \item Now Goldfarb actually knows something about Frege and Russell, so he can bring in that background into this course.
    \item Who is Wittgenstein? Wittgenstein was born in 1889 into one of the wealthiest families in Hungary - his father controlled part of the steel trade. The level of wealth was on the same scale as Andrew Carnegie. They lived in a ``palais", some grand house in Vienna, to which Brahms was a regular visitor. In 1908 he went to England to study aeronautical engineering. Around the time he was doing this engineering at Manchester, he picks up Foundations of Arithmetic and Russell's work. He asked Frege how he should study that material, who told him to go study with Russell in Cambridge. Officially, he enrolls there in the start of 1912. Russell initially finds him rather offputting and calls him ``my german". By the summer of 1912, Russell has become quite impressed and has lunch with Wittgenstein's sister, telling her that the he expected the next big step in philosophy to be made by her brother. Initially, Wittgenstein was very into the Russell paradigm.
    \item There is some literature about the influence of the Frege literature versus the influence of the Russell literature. Goldfarb doesn't think that Frege had too much influence, because he doesn't seem to really understand Frege. Wittgenstein seems very influenced by the Russellian view of propositions, etc.
    \item In the summer of 1913, he decides to go off to rural Norway to work on his own ideas. Then he comes back to Cambridge and tells Russell that he wanted to dictate his ideas to him (``Notes on Logic"). Goldfarb isn't yet settled on how much we're going to look at the notes on logic.
    \item Then he goes back to Norway to some hut and gets visits by G.E. Moore. He tells Moore that he also is going to dictate notes to him (``the Moore notes"). We might do some of that background. This deals with questions like: "are there negative facts".
    \item Then he is back in Vienna and ware breaks out; he joins up, serves on the Eastern front and then the Italian front. He then gets taken prisoner in Italy into some place in central Italy that also happened to be a big repository of medieval documents.
    \item It was published in 1921 in German, and then translated by Ogden and Ramsey with an introduction by Russell in 1922. The Ogden/Ramsey translation is not liked by Goldfarb, but it does have a sort of probative value because it was looked over by Wittgenstein. But it is still written in somewhat old-fashioned English. The preferred translation was published in the 1960s by McGuiness, use that one.
    \item Tractatus is this aphoristic book that has some structure imposed by the numbering. The arguments are never fully articulated (if there are arguments). In the preface he says that the thoughts will only be understood by those who have already have them. He had a famous letter to Bertrand Russell where he tells him to "think for himself".
    \item It has two aspects of historical influence: one that has to do with mathematics and logic, and then second is something that has been completely misconstrued within analytic philosophy.
    \item We have made progress in decoding this book, but if you read it there is something about it that we haven't quite figured out in the last 100 years. Goldfarb optimistically think shtat we've made more progress on the Tractatus in the last 20 years of philosophy then in any other field of philosophy.
    \item The chief difficulty is that the basic tenants of this book cannot be said. That is, the basic tenants of the Tractatus lie beyond the intelligible.  This is where the ``kicking down the ladder" comes from. Goldfarb thinks that what he means that when you kick away the ladder, you fall; but this is debated.
    \item \textit{Pre-tractarian} means before the Tractatus. This includes notes on Russell, Moore, and logical form. There is also the proto-tractatus which is very similar in nature to the Tractatus but is only slightly different in somewhat unimportant ways. There is a question about when the proto-tractatus was written. There is one view that dates it earlier, and another that Goldfarb didn't specify. There's this whole question of how much Frege did Wittgenstein incorporate into this work; originally Goldfarb thought there was a lot, but now he thinks not so much.
    \item He was part of a Frege reading group, and if one view of when the proto-tractatus was written says that it was written before this reading group. This is evidence that Wittgenstein was much more in the Russellian framework than in the Fregian one.
    \item His first question is about the nature of logic. How is logic possible? In a very early letter to Russell (this is during a school vacation when he was back in Vienna), he said that logic must be of a totally different kind than any other science. This is a very different view than that held by Russell, who thought that logic was just another science. ``Logic must take care of itself", he says.
    \item In those early documents, he is responding to Russell's difficulties in categorizing logic. Frege didn't worry about this; he just thought that there was logic, you could take it or leave it, and if you left it then you are insane. Russell was obsessed in discovering what the characteristic marks of logic were. 
    \item Then he gets to what is the nature of language. How can language work so that there can be logic? How must the world be so that language can describe it intelligibly? This question is very old, thinking about how must the world be so that we can sensibly talk about it.
    \item These two questions are separated only with a little distortion - he thinks that these are really the same question. ``My whole task consists in explaining the nature of the proposition; that is to say, giving the nature of all facts inside the proposition".
    \item In Goldfarb's view of Russell, he thinks Russell has an object view of subjectivity - that there are objects available to all of this. The picture theory of language is moving away from this. This is \textit{linguistic idealism}, whereby what is given to us objectively is discourse and discourse is the framing element.
    \item So, he starts with logic and in 1913 he writes something unknown to Frege that was important for everyone: the truth-table analysis of sential logic. Why do we call ``Emerson is or is not in Harvard Yard" a logical truth? Becuase any assignment of truth values to its constitutent parts yields a truth. This gives us a general story of sentential logic. It suggests to Wittgenstein that you can extend this to all logic, and it suggests the following remark about the status of the truth of logic: \textit{that they do not have any factual content}. They are truths regardless of how the world actually is, and so they have no content. But we can argue about this; we'll get into this.
    \item He fulfills what he said to Russell, that logic is a totally different science than the other natural sciences. However, this basic insight, which he expresses in various ways, like that the logical particles are not names of anything (his ``primary idea"). Frege and Russell thought that ``or" names some logical object and that the truth conditions on the connectives reflect something true about those logical objects. In a biography about G\"odel, Russell notes that G\"odel thought there was universal OR stored in the heavens (this was exactly Russell's own view at some point).
    \item Frege and Russell would say that ``Emerson Hall is or is not in Harvard Yard" is true because of its following of the proposition of the form $\forall p(p \vee \neg p)$. The truth of the form er sentence depends on its structure as inheriting from the more general law. Wittgenstein thought that only the former (Harvard Yard) is a logical law, and it is such because it is true always. You have to go through the book until you get this at 4.0312. Once you udnerstand that, you can develop the rest. But Wittgenstein doesn't want to spare us the trouble of thinking for ourselves, so he buries it at four decimal places.
    \item Let's think of a logical statement like $p \vee q$. What Wittgenstein takes is that the truth table for this statement is a better notation then the statement itself, because the logical constants ($\vee$) isn't represented at all in the truth table. Everyone took this to be correct in the 1920s and 30s - that the laws of logic have no content. That enables some of his successors to say that this is how we reconcile our happiness with logic and mathematics with empiricism (which was held by the logical positivists). Mathematics was the haven of the Rationalists. Hume didn't want rational abilities to perceive general laws, but he couldn't figure out what Wittgenstein did. By noting that mathematics and logic don't have any content, you can reconcile them with empiricism. Wittgensteinhimself is not an empiricist, and so he wasn't really interested in this. But to his empiricist predecessors this was the great virtue of his work.
    \item  One notion from Frege is that logic frames content. That you only get an idea of the content of an idea if you follow what content it follows from and what content follows from it. If you follow this thought, you conclude that logical statements just say the same thing. Frege didn't want to say this, but Wittgenstein bit the bullet and made this conclusion. 
    \item Wittgenstein thought that logic bounds what is rationally thinkable. He's focused on the bounds of discourse.
    \item see Russell's introduciton: "in order to draw a limit to thought, we should have to find both sides of the limit thinkable. It will only be in language where this limit can be found\ldots". This is part of the \textit{linguistic turn} in philosophy, whereby you have to think about language foundationally. Wittgenstein in thinking about this foundational logic finds many issues with Russell and Frege, Goldfarb isn't saying that you shouldn't think of Wittgenstein's responses to Frege as wrong, but what he rejects in the outset is Russell's view of propositions (``possible facts"). Wittgenstein thought that there are positive and negative facts, but not true or false facts. He seems to believe in the notes on logic that there are negative facts, but no longer so in the Tractatus. A negative fact is just the absense of the fact. Goldfarb does believe that the ontology of the Tractatus does not permit negative facts.
    \item Frege thought that propositions name a truth value, or names of complexes. Wittgenstein thought this was false (``especially false"). At this time Russell was also rejecting this sort of metaphysics. Russell had a different metaphysics by 1912 that Wittgenstein also found hopeless. Wittgenstein scholars prefer to translate \textit{Der Satz} as ``sentence", but in German it can also be translated as ``theorem" or ``proposition". But Wittgenstein is (according to Goldfarb) that Wittgenstin is using something closer to ``sentence"---things you write down---not propositions in the Russellian sense of abstract metaphysical objects or what not. If you read Russell during this time he seems to be gravitating toward a linguistic idea. Russell has this tendancy that he would give up parts of views that he had in the past but not fully (similar to Prof. Putnam). Goldfarb thinks that he is being very linguistic when he talks about Satz. Quine read Russell as thinking linguistically (a variable in a proposition is just a letter), but Russell originally thought that variables were some sort of entity. Maybe Goldfarb will make you read the Theory of Judgement papers from 1913, but he probably won't because he doesn't have a settled view about them.
    \item Wittgenstein's Satz are something like sentences in our language together with our full analysis (so not just sentences on the page). Or the sentences that we would arrive at if we were to conduct a full analysis. For Goldfarb's first Wittgenstein paper he was grilled on what is a Satz; he had no idea and only stumbled through it when asked at APA "smoker" meeting.
    \item What Wittgenstein calls the ``bipolarity" of sentences is fundamental to a lot of Wittgenstein. We can have sentences with sense and sentences without sence (e.g. sentences of logic which have no content). A sentence with sense is possibly true or possibly false. Or, as Goldfarb likes to put it, the content of a sentence is the difference between the sentence being true and the sentence being false.
    \item In 5.5563, "all the propositions of our everyday language just as they stand are in perfect logical order". He also says that tacit conventions on which our ordinary language resets are enormously complicated. How you can holds those thoughts in your mind, Goldfarb doesn't know. Clearly, we need to get at the tacit conventions, so it's not that our ordinary sentences are somehow misbehaving or out of order. You don't have to go far to find sentences that are not in logical order ("nothing nothings"), so you have to be delicate when thining about what he means by saying that our sentences are in perfect order.
    \item He sees philosophy as being a clarificatory activity, where we take statements in ordinary language and try to clarify it. Goldfarb really thinks he means it. Is there a doctrine in the Tractatus? Goldfarb wants to say no. We will talk about this a lot because Goldfarb has been involved in this debate for 50 years. There  are the ``austere" and ``non-austere" versions. The non-austere folks ``chicken out" by not accepting that philosophy as nonsense. The resolute interpretation strictly follows what he says, that philosophy makes no sense quite literally. For a long time this was called the ``Tractatus Wars".
    \item Try reading with just 1 digit expanded, 2 digits, 3 digits, etc. You will learn a lot by doing this, although there still is a lot buried in the fourth decimal place.
    \item In 2.1---``we picture facts to ourselves"---until 4.1 is about language, and then he doesn't start thinking about language until the mid 4s. The order is ontology, language, and then logic, which is in the reverse order or Russell/Frege.
    \item It seems like the opening of the book is making ontological statements, but if you reread that opening section after considering the importance of logic, then you'll see that it isn't really ontological. He's talking about objects, substances, the case, etc. Originally people thought that what he was talking about was sense data, but this is bizarre. That was the predominant reading of this over the first 30 years after its publication.
    \item The first impact was the logical truths have no content. The second is that people thought it was about sense data. This gave rise to Carnap, etc. version of the \textit{verification principle}, whereby a sentence is meaningful iff there is a process of verifying it.
    \item Let's go over the opening of the book. Next week we'll talk about his critique of Russell, Goldfarb will put up some readings. 
    \item He starts with ``the world" but doesn't define what is it. He wants to give you a picture, but is relying on your intuition of what the world is. Every good philosopher does this, starting with ordinary language and building off of that. He wants to start with an intuitive concept of the world. He also talks about what is and what is not the case in an intuitive way, he's not making explicit what this comes to mean yet.
    \item His math background was largely computational. The abstract mathematical objects that we talk about in pure math was just getting started, and so the mathematics he was doing was very computational. So if you consider his philosophy of mathematics, he doesn't seem to understand modern mathematics at all (groups, fields, etc.). He knew differential equations because he had to get his kite flying. The remarks on the philosophy of mathematics was edited down to shreds; Goldfarb sees reviews that say Wittgenstein didn't know any math. He did know more serious math, but Goldfarb suspects that iwas more old-fashioned stuff.
    \item What does he mean by ``logical space"? Maybe it has something to do with how the world divides. And then he further glosses over ``divides" in the next sentence: "each item can be the case or not the case while everything else remains the same". But there is not German word for ``item"; the other translation uses ``one". This is saying that the world divides into bits where each bit is independent from one another. We still don't understand exactly what is driving that, but we hav ea much better understanding than we did 10 years ago.
    \item \textit{Sachverhalten} (2), means something like states of affairs or an arrangement. He made this word up. Ogden and Ramsey call it ``atomic facts", but Goldfarb liks McGuiness better because atomic facts seem to imply that something holds, but whether or not they hold or do not hold is left up in the air for a while.
    \item Objects come in on 2.01. Here, Goldfarb thinks he's trying to tell us that he's not trying to introduce a technical term, that he doesn't really care what you call them. Then he goes on for 6 more remarks about things; we don't really know what things are still, just that they make up states of affairs.
    \item Goldfarb once attended an Alonzo Church lecture on Frege. Church was a very populist man. Goldfarb remembers him saying that Frege has ``objects" and ``concepts" very slowly as he wrote it on the board. This took up the first 15 minutes of lecture and Goldfarb left because it was moving too slowly. When W. talks about combinations of objects, he seems to be saying something very similar to what Frege was saying. Wittgenstein writes to Russell immediately after going to talk to Frege, saying that there was no logical copula, that all there is is this possibility of combination.
    \item He's using ``essential" very pointedly, because there are parts of objects that aren't essential in that they are not tru ein all possible worlds. This is the ontological rendition in the 2.0s of the context principle ("only in the context of a proposition does a word have a \textit{bedeutung}").
    \item We may move to Fridays some days because he wants Rickett's to come and talk and he's only available on Fridays.
    \item 2.0122 - Goldfarb thinks he's kind of riffing on Foundations of Arithmetic here, saying that Frege said that objects have some sort of priority of concepts but that he had no right to do that. Don't privilege one over the other, Wittgenstein seems to want to say.
    \item Goldfarb isn't going to get to Wittgenstein's coding of some anti-Kantianism, which comes just a couple of sentences later.
\end{itemize}

\section{September 13, 2017}

\begin{itemize}
    \itemsep0em 
    \item \g startied publishing on the Canvas site at 3:00pm, you clearly shouldn't have read it by now. He has things to say today that doesn't require a lot of stuff (going over the basic layout, which he did about half of last time), What he has put up on the website so far is very minimal, the key sentences of the \T (0 decimal points), and then the 1 decimal points. \g hasn't put up what he has on the 2 and 3 decimal points.
    \item If you read the 7 sentences, then yes, you will see that is the heart of what 
    \item The most important secondary book to read is Anscomb's book, she understood before anyone else the heart of this book. You'll read it and see that she kind of had the right idea, but just says it in a very obscure way. If you go back to it with the knowledge we have now, she actually is sort of on to something. She was the first person to really come out militantly for the idea that you would have to read \w against the background of Frege. She rejected that it was some kind of text about sensory things or phenomenology. But 10 years after the \T \w was a phenomenalist.
    item \g put the link up today to the PDF with the original German and both translations side-by-side. He doesn't suppose that most of you speak German, but \g finds that the Ogden/Ramsey translation isn't great, but that because \w read it and commented it is of some scholarly interest. He'll also put up Mounce's book (somewhat introductory) on \T. But a lot of what we're going to be doing is looking at the recent secondary literature on the topic.
    \item There's a book on the ``war" that is the \T called ``Throwing Away the Ladder". The book ``Beyond the Tractatus Wars" is kind of the next volume of this epoch, it is trying to say that maybe we should move on from this (but doesn't necessarily achieve that). But all the discussion of the wars is to come.
    \item What we're going to start on next week (he'll try and put this up tonight) is a start on the criticisms of \w  on Russell, from whom he draws a lot seemingly. Everything is on the Files section of the Canvas website. We'll start on this more ``advanced" topic last week. \w says that the problem with Russell's theory of judgement is that it doesn't prevent us from juding nonsense. That's all he said. Naturally, people started asking ``what is this criticism and why is it so important for him". We'll get there next week (\g hopes but cannot promise) that he'll throw it up online tonight.
    \item \g thinks of \T as a reaction to Frege and Russell, not just as a brilliant book. There are two trains of thought about this: one that (Peter Guiche) is to say that is just an evolution on what Frege is saying, but this was wrong. \w was much more working in the Russellian framework, and misunderstood Frege completely. He is reported as writing to Frege, ``Now that we have Russell's theory of descriptions, don't you agree that we no longer need the distinction between sense and reference." That's a very fine-grained scholarly matter, but the point is that with a book that is this hard to understand, anything we can do to get to the bottom of where he is coming from will help.
    \item Jacob Rosen is running a German reading group, they are reading Foundations of Arithmetic and got through only through 7 sentences this week. It was great fun to do that, according to \g.
    \item There isn't a whoel lot of difficulty with the translation of this text, but every so often you get to a place where if you know German than you will get something more out of it.
    \item \g said he was going to give a Kant reference. He referneced something about substance and change. Everyone talks about this stuff since Aristotle, so whether or not \w had a deep engagement with the canonical texts of standard philosophy before Frege in \w. It is almost if he becomes fascinated by Frege and thought that this was going to destory all of traditional philosophy. So he goes to Frege and ask him how to do this, and Frege was old and tired and told him to go speak with Russell. He thought there is a good way of doing philosophy, and that this way is going to be deeply destructive of the bad way of doing philosophy.
    \item There is a view that he is somehow channeling Shopenhauer. To Goldfarb's eye, this is wrong. what he is trying to do trying to destroy the whole structure in which we make Shopenhaurian noises. He talks about spatial spectacles (that the spatial form of things as they appear for us are due to the way we perceive them, not as they are as things in and of themselves the spatial form of things as they appear for us are due to the way we perceive them, not as they are as things in and of themselves). \w is very concerned with multiplicity. 
    \item The procedure of \w iwas to write things down adn then re-arrange it. When he had a staff he would have them cut things up and organize them. He did something like that for this book (splicing things together). He has some notes from Russell and Mooore and he had some notebooks. But this was during the war, where he was going around Europe. He was in and out of the Front; the final version was written in hist final summer in Vienna before becoming a POW. He complains to Russell that he didn't have a copy of the book and asks him for a copy. Then Russian tells Keynes to get \w a copy in the POW camp in Italy. The book was finished by then,; he had perfect access to the Frege and Russell documents that he was interested in. There is no evidence that he was reading anything else.
    \item In this book, \w doesn't talk about culture or reflecting on his cultural background because he thinks that logic is going to solve all of our problems.
    \item \textbf{2.0122 - 2.0131} is maybe where there is a connection to Kant. Kant talks about substance and change but his ideas were not necessarily new.
    \item We didn't get very far last week, we are not going to get very far this week, last week we eneded by talking about objects that are inalterable, persistent, independent of what is the case, and has no properties except what other objects it can be combined with. Possibility is a big deal. Frege has concepts nad objects, \w is kind of making fun of that because they are completely symmetrical - some things go together and some things do not. Frege is essentially thinking about mathematics and what we need in mathematics (properties of numbers instead of just numbers). He's so basing everything on the language of mathematicas, and \w said that all we have in the nature of language itself is that some things can combine with other things and other things cannot, and that we can't unpack that part of language from the outside.
    \item We have these thigns called objects, and it was a hooror game for many years of the question of ``what are these objects about \w speaks." They are \textit{whatever we need to get to the bottom of analysis}, and exactly that.
    \item One of ththings that \w ontology demands is that certain states of affairs are logically independent . How can you analyze things like "red" and "green" need to be analyzed further, but he ultimately gave up on that analysis (don't use relations in analyze them). So he gives up on \T.
    \item \w doesn't ever express in the late period the kind of doubt that he expressed in the middle period (``I used to think X but I was wrong"). \g thinks that he just didn't believe it. He had a different view of what analysis was, and in this view he couldn't figure out how to solve the problem of something being red versus something being green. The later view is much kinder to the early view than the middle view was.
    \item He says all this stuff in about objects not making up the composite of the world. Notice that he's givin gyou an ontological argument that depends on language. He says, "if we have simple objects\ldots, then whether a sentence had sense would depend on something else and wouldn't be able to sketch out a picture of the world." So how does language get off the ground, how is it even possible. So language really takes the center stage, and that's really important for all of this work.J
    \item In 3.23, "The requirement that simple signs be possible is the requirement that sense be determinate." This is clearly a linguistically based idea. Then in the proto-tractatus he makes the linguistic argument that we don't know what we're going to get to with the analysis but that the result of the tractatus is whatever we get to with our analysis. We have no familiarity with the sorts of things that are like \w simple object, they literally have no properties. Is he sketing out some really hyper-realistic view of the world? The more traditional view was anunciated by David Pears which is that \T is a hyper-realistic picture. But how do you get this picture from a linguistic argument? Pears thinks that this is what is wrong with the \T, \g isn't so sure. Brian McGuiness, who is turning 90 next month, was one of the first people to come up with the idea that the simple objects are compeltely different than anything we can get our heads around. He and others started making noise about realism: this isn't realistic, we have no conception of these objects except whereever we get with the Tractatus. If there is an argument here, there is something about how language has to work. This is something that \g has belabored in his talk of Frege, but it's the idea in order to proceed it has to do so pre-suppositionlessly. The propositions about language have to contain within themselves everything that distinguishes the claim it makes against any other conceivable world. \w largely says this, Frege more or less says this, but then \w criticizes this.
    \item We hav ethe metaphore of a proposition reaching through logical space. What is logical space? It's not worht spending to omuch time on. \w has some vague idea of what logical space, but \g thinks that there isn't such a thing. The propositions bring the structure, they don't get shoved into an existing one. \g 's senior tutor wanted to find a mathematical structure to describe the \T, but this was a bad idea.
    \item If you compare \w 's talk of substance with Aristotle's talk about substances as it changes through time, \w is talking about substance as change through different possible worlds. But \w and Frege doesn't think that change through time is actually change, two different facts "X at 2:00pm" and "Y at 3;00pm" are just different facts, there is no change they are just propositions that have enough indexed points. You just have propositions, they are different facts. what he is doing is kind of referencing substance in Aristotle, even though you can understand what he is saying without it. He is giving the more modern, logical version of what Aristotle is saying about substance. He's just trying to rope you in. We don't have a whole lot of documents, so it's hard to say what is rhetorical and what is logical. This isn't in the notebooks - the important parts of the notebooks is 1945/1915. \g rejects that the way to read the \T is to reference his notebooks, he is developing and changing through these years.
    \item \w uses substance purposefully drawing on the proposition. But it is not what changes/persists through time, it is about what is common between possible worlds. You also know that he doesn't care that much about substance because he drops using the word ``substance" after the first few pages. This is like Frege dropping the references to analyticity in the first few pages.
    \item We only understand stuff by talking about what every conceivable world has in common in our world. Don't read \g as saying that \w is a linguistic idealist, he is not saying that. It is not ``an imagined world" but ``a world that we can conceive of".
    \item This is a very concise book. Dave Stern said it is an analytic table of contents in search of a book. It is a very condensed text and yet there are a lot of reptitions in it. There are very slight difference in elocution of them.
    \item The idea of pre-suppositionlesness is clear in this book. We don't want our propositions to depend on the truth or falsity of particular complexes. Russell had this notion that complexes couldn't be analyzed (``the complex is red", etc.) if the complex doesn't exist then it didnt' have a truth value. Against Frege's notion of pre-supposition, then the content of the proper names must mean something (``Kepler died in misery."). If the name ``Kepler" did not refer, then the epxression didn't have a truth value. Frege gets in great difficulties about this. \g thinks thatif a sentence doesn't have a truth value, then we can't apply logic to the sentences. How can it be that you have to know facts in order to apply logic, it cannot be the case that you have to know facts in order to know logic. That remark may also be working against later Russell and reminding us about Russell's theory of descriptions where you always have to analyze complexes and get a truth value (By replacing ``Kepler" with a sufficiently complex desription that has a definite truth value).
    \item Use ``sentence along with its full logical analysis" or just ``sentence", don't use ``proposition."
    \item ``What a description of reality really is\ldots" - in the notebooks, he says that the method of portrayal must be determined even before we already being. That's what the insight is here, it would be nice if he had been more cleary about this in the \T but that is true of a lot of things in there.
    \item You can see that if you have a view of modern logic where if you think that there has to be a great distinction between logic and empirical matters, then you'll be pulled in a completely different direction. The underlying notion is that unless claim scan be fully articulated without specifying the assumptions, then we cannot get at the heart of rational discourse. This is precondition for there being such a thing as rationality. He just wants the basic commitmtne tof to what we must have in order to be rationality. Reality has to be fixed to a sentence to get a truth value. 
    \item ``Even if we have no acquaintance with simple objects but do know complex objects by acquaintance\ldots" - it's clear that we only know that the simple objecst are the end product of analysis, we don't know anything else about them. 
    \item He starts in 2.1 talking about pictures. In the proto-tractatus he says that ``we conceive of (grasp) of facts as pictures". Why is this different in the final version? He wanted to be more vague about it.
    \item Don't think that you have some conception of reality and then you talk about how we depict reality against that background, really think about that we get to reality through how we depict it. He thinks that we do get something about the world through the structure of language, but that language has an intrinsic logic to it as well (?? XX shitty transcription).
    \item He surprises us with what he talks about with there being a basic level at which things are logically independent. Therefore you know that relations likee "on" cannot be primal. The question is that once you get down to that level, once you get down to the logically independent things, are they going to have to be self-interpreting.
    \item The picture of the cat on the mat is a spatial picture of a spatial situation. We also have the picture of music bars at teh beginning of the tune. That one note comes to the right on the sheet music of another map depicts as a picture that the note comes after the former in the music.
    \item In 2.15, the usual translation is there is a commonality in form between the picture and the depicted. \g doesn't think that he believes that there is a bustantive relation between the picture and the depicted, rather that this is a logical relation. It's not that we have a substantial commonality ("after" versus "to the right") but rather that there is some kind of logical structure shared between the two. We have tthis sort of logical multiplicity and that is going to be the same as some sort of logical multiplicity that is out there in the world. We have two issues here: what does he mean by commonality? \g thinks that it really it Fregian, that there isn't a substantive relation. The second question is what is the argument for there to be anything like commonality. 
    \item The difference between vertreten and darstelle is the difference between the pope in rome and the guy in the passion of the christ (?). We have this conventional proxy relation, but this cannot be the whole story. What makes the picture of sheet music represent the music? A picture is not just a jumple of elements, that it's a combination of parts - rather there is something that holds about the sheet music that is shared with the music. This isn't the correlation of elements, but it's got to be a common something. There has to be a stopping point that isn't going to be a correlation or identity. It is not a substantial relationship. Although some pictures do share a common relationship, most pictures don't have the same relation as the cat on the mat diagram. What \g is trying to say is that what is identical between them is not a pictoral relation but rather the logical glue.
    \item How does the picture depict reality? It reaches out to us. Maybe \w is giving us a hint that we aren't actually going to figure out the theory of representation here.
    \item The notion of form he has (the notion of possibility reaappears here) - he says that form is the possibility of structure.
\end{itemize}

\end{document}
