\documentclass[12pt]{article}
\usepackage{latexsym}
\usepackage{amssymb,amsmath}
\usepackage{amsthm}
\usepackage{mathtools}
\usepackage[margin=0.75in]{geometry}
\allowdisplaybreaks
\setcounter{tocdepth}{4}
\setcounter{secnumdepth}{2}

\theoremstyle{definition}
\newtheorem{theorem}{Theorem}
\newtheorem{defn}{Definition}
\newtheorem{prop}{Proposition}
\newtheorem{cor}{Corollary}
\newtheorem{lemma}[theorem]{Lemma}
\newtheorem{exmp}{Example}[section]
\newtheorem{remark}{Remark}

\usepackage[T1]{fontenc}
\usepackage{ccfonts}
\setlength\parindent{0pt}

\begin{document}

\begin{center}
\textbf{Math 155R: Combinatorics} \\
Alexander H. Patel \\
{\tt alexanderpatel@college.harvard.edu} \\
Last Updated: \today
\end{center}

These are lecture notes for the Spring 2017 offering of Harvard Math 155R:
Combinatorics. Pardon any mistakes or typos.

\tableofcontents

\section{January 23, 2017}

\subsection{Enumerative Combinatorics}

The subject matter of the course is enumerative combinatorics. In general,
combinatorics deals with finite structures; enumerative combinatorics primarily
deals with counting (the cardinality of a particular set).

\begin{exmp}
    Take $S = \{1, \ldots n\}$, $T = \{1, \ldots m\}$. How many maps $f: S
    \rightarrow T$? There are $n^m$ possible maps.
\end{exmp}

\begin{exmp}
    Take $S = \{1, \ldots n\}$. How many of the possible maps $f: S
    \rightarrow S$ are permutations? There are $n!$.
\end{exmp}

\begin{exmp}
    How many of the $n!$ permutations in the last example are
    derangements (i.e. $f(i) \ne i \forall i$)? One way to solve is to
    find the probability that any particular permutation is a
    derangement. 
\end{exmp}

\begin{theorem}[Fermat's Little Theorem]
    Let $n$ be an integer and $p$ be a prime. Then $n^p \equiv n \mod
    p$ (in other words, $n^p - n$ is divisible by $p$).
\end{theorem}

\begin{proof}
    Let $S = \{1, \ldots, p\}$ and $U = \{1, \ldots n\}$. Let $X = \{f:
    S \rightarrow U\}$. Arrange every element of $S$ in a cycle, and
    then compose any function from $X$ with a rotation on the cycle.
    You then have the map $T: X \rightarrow X$, $T: f \rightarrow f
    \circ r$. If you compose with $r$ $p$ times, then you get back to
    where you started.

    Divide $X$ into 2 subsets. Let $X_0$ be the constant functions (all
    elements of $S$ get sent to a single element in $U$). $\#X_0 = n$.
    Let $X_1$ be the non-constant functions. For $f \in X_1$, the maps
    $\{f, Tf, T^2f, \ldots T^{p-1}f\}$ are all distinct. $X_1$ is a disjoint
    union of subsets of size $p$. This means that $p$ divides $\#X_1$, but we
    know that $\#X = n^p$, and $\#X_0 = n$, and so $\#X_1 = n^p- n$ and so $p |
    n^p - n$.
\end{proof}

\subsection{Graphs}

\begin{defn}
    A graph $G$ is a set $V$ ("vertices") and a subset $E$ of the set of
    two-element subsets of $V$ ("edges"). There are two extremes: say $\#V =
    n$. Could have $E = \emptyset$ or $E$ consists of all two-element subsets
    of $V$. 
\end{defn}

\begin{theorem}[Ramsey's Theorem]
    Given $k > 0$, $\exists n$ such that $\forall$ graphs $G$, if $\#V = n$
    then $G$ contains either an $L_k$ ($k$ vertices with no edges) or a $K_k$
    ($k$ vertices with edges between all of them).
\end{theorem}

\begin{exmp}
    Given a $2 \times n$ array and $n$ dominoes, where a domino is a $2 \times
    1$ array. You want to cover the array with the dominoes: how many ways are
    there of tiling a $2 \times n$ with $n$ dominoes?. 

    Let $T_n$ be the number of ways to tile. Note that $T_1 = 1, T_2 = 2, T_3 =
    3$. In the $n$ case, the uppermost left domino can either be horizontal or
    vertical. If vertical, then $T_{n-1}$ ways. If horizontal, then $T_{n-2}$.
    So, $T_n = T_{n-1} + T_{n-2}$.
\end{exmp}

\section{January 25, 2017}

\subsection{Generating Functions}

Last time: domino-covering problem. We found that $T_n = T_{n-1} + T_{n-2}$ for
$n \ge 2$. Can we express $T_n$ in closed form? Yes, using the technique of a
\textit{generating function}. Introduce $F(x) = \sum_{n=0}^\infty T_nx^n \in
\mathbb{Z}[[x]]$. Don't think about $F(x)$ as a function, think about it as a
power series. It will converge for some values, but not all of them - so just
do the calculations in the ring of formal power series. 

\begin{align*}
F(x) &= \sum_{n=0}^\infty T_nx^n \\
 &= 1 + x + \sum_{n=2}^\infty T_nx^n \\
 &= 1 + x + \sum_{n=2}^\infty(T_{n-1} + T_{n-2})x^n \\
 &= 1 + x + \sum_{n=2}^\infty T_{n-1}x^n + \sum_{n=2}^\infty T_{n-2}x^n\\
 &= 1 + x + x(F(x) - 1) + x^2F(x)\\
 &= 1 + xF(x) + x^2F(x) \\
 (1 - x - x^2)F(x) &= 1 \\
 F(x) &= \frac{1}{1-x-x^2}\\
\end{align*}

Next, write out the power series expansion of $F(x)$. Let $\phi = \frac{1
    +\sqrt{5}}{2}$. Then, the roots of the denominator of $F(x)$ are $-\phi,
    \phi - 1 = \frac{1}{\phi}$ - i.e. $(1 - x - x^2) = (1 - x/\phi)(1 + \phi
    x)$. So, 

\begin{align*}
F(x) &= \frac{1}{1 - x - x^2}\\
&= \frac{\lambda}{1 - x/\phi} + \frac{\mu}{1 + \phi x}\\
1 &= \lambda(1 - \phi x ) + \mu(1 + x/\phi)\\
\lambda &= \frac{1}{1 + \phi^2}, \mu = \frac{\phi^2}{1 + \phi^2}\\
\frac{1}{1-x-x^2} &= \frac{1}{1 + \phi^2} \frac{1}{1 - x/\phi} +
\frac{\phi^2}{1 + \phi^2} \frac{1}{1 + \phi x}
\end{align*}

Since $\frac{1}{1-x} = 1 + x^2 + \ldots$, we can re-write $\frac{1}{1 - \phi
x}$ as $\sum_{n=0}^\infty\phi^n x^n$ and $\frac{1}{1 + x/\phi}$ as
$\sum_{n=0}^\infty (-1\phi)^n x^n$. 

So, then just equate the coefficients from the two ways of expressing $F(x)$:

$$T_n = \frac{1}{1 + \phi^2}(-1/\phi)^n + \frac{\phi^2}{1 + \phi^2} \phi^n$$

\subsection{Stirling Numbers}

\begin{defn}
A \textbf{partition} of a set $\Sigma_n = \{1, \ldots n \}$ is an expression of
$\Sigma_n$ as a disjoint non-empty union of subsets. In other words, we have an
equivalence relation ($a \equiv a, a \equiv b \Leftrightarrow b \equiv a, a
\equiv b \& b \equiv c \Rightarrow a \equiv c$) on $\Sigma_n$.
\end{defn}

Question: How many partitions of $\{1, \ldots, n\}$ with $k$ parts are there?
These are denoted $\begin{Bmatrix}n\\ k \end{Bmatrix}$ and are called
\textbf{Stirling numbers}.

\begin{itemize}
    \item $\begin{Bmatrix} 0 \\ k \end{Bmatrix}$ is 0 if $k > 0$ or 1 if $k =
        0$ (mostly because of convention). 
    \item $\begin{Bmatrix} n \\ k \end{Bmatrix} = 0$ whenever $k < n$ 
    \item $\begin{Bmatrix} n \\ 0 \end{Bmatrix} = 0$ is 1 if $n = 0$ and 0 otherwise.
    \item $\begin{Bmatrix} n \\ 1 \end{Bmatrix} = 1$ if $n  > 0$ and 0 if $n = 0$.
\end{itemize}

Next, to find a closed-form expression of a given Stirling number for $\{1,
\ldots, n \}$. Ask: is $\{n \}$ one of the subsets? Break the partitions in
which $\{n\}$ is a subset and those partitions in which it is not. If "Yes",
then just solve $\begin{Bmatrix} n - 1 \\ k - 1\end{Bmatrix}$. If "No", then
$\begin{Bmatrix} n - 1 \\ k \end{Bmatrix} * k$.

Conclusion: $\begin{Bmatrix} n \\ k \end{Bmatrix} = \begin{Bmatrix} n - 1 \\ k
    - 1\end{Bmatrix} + \begin{Bmatrix} n - 1 \\ k \end{Bmatrix} * k$

\section{January 27, 2017}

\subsection{Sterling Numbers}

Last time: $\begin{Bmatrix}n \\ k\end{Bmatrix}$ is the number of partitions of
the set of $n$ elements into $k$ non-empty subsets (also the number of
equivalence relations on the set with $n$ elements with exactly $k$ equivalence
classes). 

\begin{exmp}
    $\begin{Bmatrix} 4 \\ 2 \end{Bmatrix} = 7$. You have $\{1, 2\} \cup \{3,
    4\}, \{1, 3\} \cup \{2, 4\}, \{1, 4\} \cup \{1, 3\}, \{1\} \cup \{2, 3, 4\},
    \{2\} \cup \{1, 3, 4\}, \{3\} \cup \{2, 1, 4\}, \{4\} \cup \{1, 2,3\}$.
\end{exmp}

We want to solve the recursion relation derived at the end of the last lecture
with a generating function. Fix $k$ such that: 
\begin{align*}
F_k &= \sum_{n=0}^\infty \begin{Bmatrix}n \\ k \end{Bmatrix} x^n \\
&= \sum_{n=0}^\infty \begin{Bmatrix}n-1\\k-1 \end{Bmatrix} x^n +
\sum_{n=0}^\infty k\begin{Bmatrix}n-1 \\ k\end{Bmatrix}x^n\\
&= xF_{k-1} + kxF_k\\
&= \frac{x}{1 - kx}F_{k-1}\\
F_0 &= \sum \begin{Bmatrix} n \\ 0 \end{Bmatrix} x^n = 1\\
F_1 &= \frac{x}{1-x} = x + x^2 + x^3 + \ldots\\
F_2 &= \frac{x^2}{(1-x)(1-2x)}\\
\end{align*}

Use partial fractions to solve $F_2$: 
\begin{align*}
    \frac{1}{(1-x)(1-2x)} &= \frac{A}{1-x} + \frac{B}{1-2x}\\
    1 &= A(1-2x) + B(1-x) \\
    2A + B &= 0, B = -2A,A + B = 1\\
    A &= -1, B=2\\
    F_2 &= x^2\left(\frac{-1}{1-x} + \frac{2}{1-2x} \right)\\
    &= x^2\left(-\sum x^n + 2^{n+1}x^n\right)\\
    \begin{Bmatrix}n \\ 2\end{Bmatrix} = -1 + 2^{n-1} = 2^{n-1} - 1
\end{align*}

Note that there are $2^n$ subsets of the set with $n$ elements, but two (the
empty set and the whole set) are not proper non-empty subsets. So then you get
$\frac{2^n - 2}{2} = 2^{n-1} - 1$.\\

\subsection{Binomial Coefficients}

\begin{defn}
    ${n \choose k}$ is the number of $k$-element subsets of the set with $n$
    elements. You could think of this also as the number of sequences of $k$
    distinct elements of the set $\{1, \ldots, n\}$ divided by $k!$ (because of
    sequence ordering).
    $${n \choose k} = \frac{n(n-1)(n-2)\ldots(n-k+1)}{k!} = \frac{n!}{k!(n-k)!}$$
\end{defn}


Find the recursion relation: partition the set of subsets of the set with $n$
elements with $k$ elements into subsets containing $n$ and subsets not
containing $n$. A subset containing $n$ amounts to a subset of $k$ elements on
the set with $n-1$ elements, The number of subsets not containing $n$ is ${n-1
\choose k}$. So ${n \choose k} = {n-1 \choose k} + {n - 1 \choose k - 1}$.

\begin{align*}
    F_n &= \sum_k {n \choose k}x^k \\
    &= \sum {n-1 \choose k} x^k + \sum {n - 1 \choose k - 1}x^k \\
    &= F_{n-1} + xF{n-1}\\
    F_n &= (1 + x)F_{n-1}\\
    F_0 &= 1\\
    F_n &= (1 + x)^n = \sum {n \choose k} x^k
\end{align*}

\begin{exmp}
Evaluate $\sum_{k=0}^n k{n \choose k}$. One way to phrase this problem:
how many chaired committees can be formed from $n$ people?
\end{exmp}

\begin{itemize}
\item First choose $k$ people for committee out of $n$ people, then choose one
    of $k$ as the chair.
\item Choose chair first and then choose subset from remaining $n-1$. So the
    answer is $n2^{n-1}$.
\item Alternatively, use a generating function (then take the derivative and
    plug in $x = 1$):
    \begin{align*}
        (1 +x)^n &= \sum_{k=0}^n {n \choose k}x^k\\
        n(1 + x)^{n-1} &= \sum k {n \choose k} x^{k-1}\\
        n * 2^{n-1} &= \sum k {n \choose k}
    \end{align*}
\end{itemize}

\subsection{Derangements}

\begin{defn}[Derangement]
Let $S = \{1, \ldots n\}$. A permutation of $S$ is a bijection $S \rightarrow
S$. We have $n!$ permutations. A derangement is a bijection $f$ such that $f(k)
\ne k \forall k$. How many derangements are there? What is the
probability that a randomly chosen permutation has a fixed point?
\end{defn}

Let $D_n$ be the number of derangements of $\{1, \ldots, n\}$ and let $X$ be
the set of all permutations (of cardinality $n!$). We want to decompose $X$
into subsets based on the number of fixed points per subset - so $X = \coprod
X_k$ where $X_k$ is the set of permutations with exactly $k$ fixed points.

It follows that:
\begin{align*}
n! &= |X| = \sum_{k=0}^n |X_k| = \sum_{k=0}^n = {n \choose k}D_{n-k}\\
1 &= \sum_{k=0}^n \frac{D_{n-k}}{k!(n-k)!}\\
\sum_{n=0}^\infty x^n &= \sum_{n=0}^\infty \sum_{x=0}^n \frac{D_{n-k}}{k!(n-k)!}x^n
\end{align*}

\section{January 30, 2017}

There is a typo in Theorem 5 (L4p3). It should be $\sum \frac{m^n}{m!}$.

\subsection{Derangements}

Recall: $D_n$ is the number of derangements on the set with $n$ elements. We
asked: what fraction of all permutations are derangements? The recursion
relation is derived by letting $\{1, \ldots, n\} = \coprod_{k=0}^n X_k$ where
$X_k$ is the number of permutations with exactly $k$ fixed points. First
specify the $k$ elements that will be fixed, and then calculate the derangement
on the remaining $n - k$ members. There are ${ n \choose k}$ such fixed point
combinations. 

\begin{align*}
n! &= \sum_{k=0}^n {n \choose k} D_{n-k} \\
\sum n!x^n &= \sum_{n=0}^\infty\sum_{k = 0}^n {n \choose k} D_{n-k}x^n \\
\sum x^n &= \sum_{n=0}^\infty \sum_{k = 0}^n \frac{D_{n-k}}{k!(n-k)!}x^n \\
&= \sum_{k,l \ge 0} \frac{1}{k!l!} D_l x^{k + l} \\
&= \left(\sum_{k=0}^\infty \frac{x^k}{k!}\right)\left)\sum_{l=0}^\infty \frac{D_l}{l!}x^l\right) \\
&= e^x\left(\sum_{l=0}^\infty \frac{D_l}{l!}x^l\right)
\end{align*}

We're summing all integers $n$ and then all integers $k$ in the range of 0 to
$n$. This is the same thing as summing over all integers $k$ and all integers
$n-k$. So set $l = n-k$ so that $n = k+l$. Also, you can write $x^{k + l}$ as
the product of two series. The result is called the exponential generating
function.

The point of the use of generating functions is to package the sequence in a
compact form instead of having to take the recursion relations one at a time.
If we want to package information, we will often be able to do so in different
ways, and the exponential generating function is often a helpful way of
packaging. Often, the exponential generating function will be identifiable but
not another generating function.

The ratio $\frac{D_l}{ l!}$ is the probability of a permutation being a
derangement. 

\begin{defn}[Exponential generating function]
    If $c_0, c_1, c_2, \ldots$ is any infinite sequence, then the exponential
    generating function of the $c_i$ is $\sum_{n=0}^\infty \frac{c_n}{n!}x^n$.
    So $\frac{1}{1-x} = e^x F \Rightarrow F = \frac{e^{-x}}{1-x}$ is the one
    for derangements.
\end{defn}

We have that:
\begin{align*}
    \sum \frac{D_n}{n!} x^n &= \frac{e^{-x}}{1-x} \\
    &= \left(\sum_{p=0}^\infty x^p\right)\left(\sum_{q=0}^\infty
    \frac{(-1)^q}{q!}x^q\right) \\
    \frac{D_n}{n!} &=  \sum_{p+q=n} \frac{(-1)^q}{q!} \\
    &=  \sum_{q=0}^n \frac{(-1)^q}{q!} \\
    &= 1 -1 + 1/2 - 1/6 + 1/24 + \ldots + (-1)^n/n! \\
    D_n &= n! - n! + n!/2 - n!/6 + \ldots + (-1)^n
\end{align*}

$D_n/n!$ is $e^x$ cut off after $n$, and note that the error between it and
$1/e$ is very small in large $n$. $D_n$ is the closest integer to $n!/e$. So
the probability that a given permutation is a derangement approaches
$\frac{n!}{e}$.

\subsection{Bell Numbers}

\begin{defn}[The Bell Numbers]
Question: How many partitions are there of $\{1, \ldots, n\}$? Equivalently,
how many equivalence relations are possible on $\{1, \ldots, n\}$? We call this
number $b_n$.
\end{defn}

\begin{remark}
The Bell Numbers are different than the Stirling Numbers because we are not
considering how many subsets the partition should have. Clearly, $b_n =
\sum_{k=0}^n \begin{Bmatrix} n \\ k \end{Bmatrix}$.
\end{remark}

Recursion relation: A partition of $\{1, \ldots, n + 1\}$ gives rise to a
decomposition of $\{1, \ldots, n\}$ into two set: the numbers $i$ in the same
part as $n + 1$ and then the rest. Let's say that there are $k$ elements in the
first set and $n - k$ elements in the second part. What you get is: $b_{n+1} =
\sum_{k=0}^n {n \choose k} b_{n-k}$.

Instead of waiting for the generating function to manifest, we just choose the
exponential generating function (so, divide by $n!$, multiply by $x^n$, then
sum).

Goal: find the exponential generating function for $F = \sum \frac{b_n}{n!}x^n$.

\begin{align*}
    \frac{b_{n+1}}{n!} &= \sum \frac{1}{k!(n-k)!} b_{n-k} \\
    F' = \sum \frac{b_{n+1}}{n!} x^n &= \sum_n \sum_{k=0}^n  \frac{1}{k!(n-k)!} b_{n-k} \\
    &= \sum_{k, l} \frac{1}{k!l!} b_l x^{k+1} \\
    e^xF &= \left(\sum \frac{1}{k!}x^k\right)\left(\sum \frac{1}{e!}b_lx^l\right) \\
    F' &= e^xF \\
\end{align*}

\subsection{Formal Power Series}

We are working with the ring of formal power series. Fix a field $K$ of
characteristic 0 $K[[x]] = \{\sum_{n=0}^\infty a_n x^n : a_n \in K\}$. These
are not functions, infinite sums do not have to exist or converge in fields. We
just treat this as a formal entity (like a polynomial) just longer. We can
carry out all the operations that we usually do with power series and
polynomials:
\begin{enumerate}
    \item add, multiply ($k[[x]]$ has structure of commutative ring with identity)
    \item If $f = \sum a_nx^n $ and $a_o \ne 0$ then $f$ has a reciprocal power
    series $g = \frac{1}{f}$. So you can divide power series under this condition.
    \item composition: $f, g \in  K[[x]]$ and $\sum b_n x^n, b_0 = 0$ then you can compose.
\end{enumerate}

\section{February 1, 2017}

Today: Bell Numbers and the theory of species (through category theory). There
are notes on categories and functors up on the course web page.

\subsection{Bell Numbers}

\begin{defn}
    $b_n$ is the number of partitions of the set $\{1, 2, \ldots, n \}$ into
    non-empty subsets. This is the same as the number of possible equivalence
    relations on the set of $\{1, \ldots, n\}$. $b_0 = 1, b_1 = 1, b_2 = 2, b_3
    = 5$. $b_n = \sum_{k=0}^n \begin{Bmatrix} n \\ k \end{Bmatrix}$.
\end{defn}

In general, there are two extremes: every element is a different partition, or
only one cell in partition with $n$ elements.

Recursion relation: say the part containing $n + 1 $ also contains $k$ elements
of $\{1, \ldots , n\}$. The recursion relation is $b_{n+1} = \sum_{k=0}^n {n
\choose k} b_{n-k}$.
\begin{align*}
b_{n+1} &= \sum_{k=0}^n {n \choose k} b_{n-k} \\
\frac{b_{n+1}}{n!} &= \sum_{k=0}^n \frac{1}{k!(n-k)!} b_{n-k} \\
\frac{b_{n+1}}{n!} x^n &= \sum_{k=0}^n \frac{1}{k!(n-k)!} b_{n-k} x^n \\
\sum \frac{b_{n+1}}{n!} x^n &= \sum_{n=0}^\infty \sum_{k=0}^n \frac{1}{k!(n-k)!} b_{n-k} x^n \\
\sum \frac{b_{n+1}}{n!} x^n &= \sum_{k, l \ge 0}^\infty \frac{1}{k!l!} b_{n-k} x^{k+l}
\end{align*}
Set $F = s\sum \frac{b_n}{n!} x^n$. It does not matter whether this converges
(it does only for some small interval about the origin). However, convergence
is relevant because if it does not converge than it is unlikely that the power
series will be expressible with elementary functions. The relation between the
expression we've derived and $F$ is that our expression is $F'$. So,
\begin{align*}
    \sum \frac{b_{n+1}}{n!} x^n &= \sum_{k \ge 0}^\infty \frac{x^k}{k!} \sum_l
    \frac{b_l x^l}{l!} \\
    F' &= e^xF
\end{align*}
Solutions to this equation are of the form $F = ce^{e^x}$. So now if you
compare constant terms, then you get $c = \frac{1}{e} e^{e^x} = e^{e^x - 1}$.
Subtracting the 1 makes sense for power series because it means there is no
constant from the differential equation.
\begin{align*}
    e^{e^x} &= \sum \frac{(e^x)^m}{m!} \\
    &= \sum \frac{1}{m!} e^{xm} \\
    &= \sum_m \sum_n \frac{1}{m!} \frac{(mx)^n}{n!} \\
    &= \sum_{m,n} \frac{m^n}{m!n!} x^n
\end{align*}
Compare coefficients of $x^n$: $e\frac{b_n}{n!} = \sum_m \frac{m^n}{m!n!}$. So,
$b_n = \frac{1}{e}\sum_m \frac{m^n}{m!n!}$.

\subsection{Theory of Species}
\begin{defn}[Species]
    A species $S$ is a device that does 2 things:
    \begin{itemize}
        \item It associates to every finite set $I$ another finite set $S(I)$.
        \item It associates to every bijection of sets $\pi: I \rightarrow J$ a
            corresponding bijection $S(\pi): S(I) \rightarrow S(j)$. 
    \end{itemize}
    Subject to the following conditions:
    \begin{itemize}
        \item $I, J, K$ finite sets. $\pi: I \rightarrow J$, $\pi': J
            \rightarrow K$ bijection. Then $S(I) \rightarrow_{S(\pi)} S(J)
            \rightarrow_{S(\pi')} S(K)$ and we require that $S(\pi' \circ \pi)
            = S(\pi') \circ S(\pi)$.
        \item If $\pi = id_I: I \rightarrow I$ then $S(\pi) = id_{S(I)}$. This
            follows from the first condition.
    \end{itemize}
    The relevant category is the category with objects being finite sets and
    whose morphisms are bijections amongst those sets. A species is just a
    functor from that category onto itself.
\end{defn}

\begin{exmp}[Species of partitions]
    Let $S(I)$ be the set of partitions of $I$ and $\pi: I \rightarrow J$.
    $S(\pi)$ maps partitions of $I$ onto partitions of $J$.  So $S(\pi) : I =
    \coprod I_2 \rightarrow J = \coprod(\pi(I_2))$.
\end{exmp}

\begin{exmp}[Species of subsets] 
$S(I)$ is the subsets of $I$ and the Specifies of
$k$-element subsets $S(I)$ is subsets of $I$ with $k$-elements.
\end{exmp}

\begin{exmp}[Species of graphs]
$S(I)$ is the set of graphs with vertex set $I$. We can
think of the set of edges as the symmetric product of $I \times I$ minus the
diagonal. Think about how to formulate the species of connected graphs.
\end{exmp}

\begin{exmp}[Species of undecorated sets]
    $S(I)$ is the set of one element for all $I$. 
\end{exmp}

\begin{exmp}[Species of non-empty undecorated sets]
    $S(I)$ is the set of one element if $I \ne \emptyset$ and $\emptyset$ if $I
    = \emptyset$.
\end{exmp}

\section{February 6, 2017}

\subsection{Theory of Species}

Last time: we introduced the notion of a species, which in its simplest form is
something that associates finite sets to finite sets and bijections between
finite sets. If you ahve category $C$ whose objects are finite sets and whose
morphisms are only bijections. A species $S$ is a functor from $C \rightarrow
C$.

\begin{exmp}
    (species of partitions) $S[I]$ is the set of partitions of $I$, (species of
    subsets) $S[I]$ is the set of subsets of $I$, (linear orderings of $I$),
    the set of permutations of $I$, derangements, graphs with vertex set $I$,
    connected graphs with vertex set $I$.
\end{exmp}

There were two trivial species: $S[I]$ associates to every set a set with one
element, $S[I]$ is the set with one element if $I$ is non-empty and $\emptyset$
otherwise. These are called the species of decorated/non-decorated sets.

What are we going to do with species? Associated to a species $S$ its
\textbf{exponential generating function}. Let $a_n = |S[\{1, \ldots, n\}]|$ and
set $F_S = \sum_{n = 0}^\infty \frac{a_n}{n!}$. In the book this is written
$F_S(x)$ to emphasize that it is a formal power series, but this wrongly
suggests that this is a function.

\begin{exmp}
    If $S$ is the species of permutations, then $a_n = n!$ and the generating
    function $F_s = \sum_{n \ge 0} \frac{n!}{n!} x^n = \frac{1}{1-x}$.
\end{exmp}

\begin{exmp}
    Let $S[I]$ associated to every finite set $I$ a one-element set. Since
    $|S[I] = 1 \forall I$, then $F_s = \sum \frac{x^n}{n!} = e^x$.
\end{exmp}

What we will be trying to do is establish a dictionary that will allow us to
move between species and power series. The basic idea is that if we have one or
more species we can combine them in certain ways that are relatively natural
and we can ask: if I know the generating function of two speices, can I
describe the generating function that occurs when we combine them?

\begin{center}
\begin{tabular}{ c c }
    Operations on Species & Operations on Power Series\\ 
    $S + T$ & $F_{S + T} = F_S + F_T$ \\ 
    $S'$ & $F_{S'} = (F_S)'$ \\
    $ST$ & $F_{ST} = F_SF_T$ \\
    $S \circ T$ & $F_{S \circ T} = F_S \circ F_T$
\end{tabular}
\end{center}

\begin{exmp}
    Define the sum of two species $S, T$ as $(S + T)[I] = S[I] \coprod T[I]$.
    Easy to note that the bijection is preserved and the cardinality is the sum
    of the species. So $F_{S + T} = F_S + F_T$.
\end{exmp}

Given $S$, can define new species $S'$ by $S'[I] = S[I\cup \{*\}]$. Then
$|S[<n>]| = |S[<n + 1>]|$.  So $b_n = a_{n + 1}$. But $F_S = \sum
\frac{a_n}{n!}x^n$ and so $F_{S'} = \sum \frac{a_{n+1}}{n!} x^n = (F_S)'$ 

\begin{defn}[Product Species]
    Say $S$ and $T$ are each species. Define the product species $ST$ as the
    species defined by $ST[I] = \coprod_{I = I_0 \cup I_1} S[I_0] \times
    T[I_1]$.  This is the same thing as the set of triples $(P, x, y)$ where
    $P$ is a partition of $I$ into two parts $I_0$ and $I_1$, $x \in S[I_0]$
    and $y \in T[I_1]$.
\end{defn}

\begin{defn}[Power Series of Product Species]
    Say $|S[<n>]| = a_n$ and $|T[<n>]| = b_n$ so $F_S = \sum \frac{a_n}{n!}x^n$
    and $F_T = \sum \frac{b_n}{n!} x^n$ so $F_{ST} = \sum \frac{c_n}{n!}x^n$
    where $c_n = |ST[<n>]|$.

    A priori, $c_n = \sum_{I = I_0 \cup I_1} \frac{a_{|I_0|}b_{|I_1|}} =
    \sum_{k=0}^n {n \choose k}a_k b_{n-k}$. So $\sum \frac{c_n}{n!} x^n =
    \sum_{n=0}^\infty \sum_{k=0}^n \frac{1}{k!(n-k)!} a_kb_{n-k}x^n$.
    Rearranging the summation and setting $l = n-k$, $F_{ST} = \sum_{k, l \ge
    0} \frac{a_kb_l}{k!l!} x^{k+l} = F_SF_T$.
\end{defn}

\begin{exmp}[Species of Derangements]
    Let $T[I]$ be the set of derangements of $I$. Our goal is to find the
    generating function for $T$. First, define $S[I] = \{*\}$ (assigns to every
    set a single element set, called in the notes the species of undecorated
    sets). If I have an arbitrary permutation of a set with $n$ elements, I can
    specify the permutation by specifying the fixed points and then the
    derangement of the complimentary set. So you can specify the permutation as
    the fixed points and the derangement. So to specify the permutation you
    have to specify the fixed point set and the derangement of the complement.
    So the product $ST$ is the species of permutations.

    We know that $F_S = e^x$ and $F_{ST} = \frac{1}{1-x}$, so $F_T =
    \frac{1}{1-x}\frac{1}{e^x} = \frac{e^{-x}}{1-x}$.
\end{exmp}

\begin{defn}[Species Composition]
    Say $S$ and $T$ are species; assume that $T[\emptyset] = \emptyset$. Define
    $S \circ T$ by $S\circ T[I] = \coprod_{I = I_1 \cup \ldots \cup I_K}
    S[\{I_1, \ldots, I_K\}] \times \prod_{\alpha = 1}^k T[I_\alpha]$. Basically
    apply $S$ to the set of equivalence classes and multiply it by $T$ when
    applied to each individual part of the partition. In other words, $S \circ
    T$ is the set of triplets $\{(~, x, {y_J}_{J \in I/~})\}$
    where $~$ is an equivalence relation on $I$, $x \in S[I/~]$, and $y_J \in
    T[J]$ where $J$ is an equivalence class.
\end{defn}

\begin{exmp}
    Take $S[I]$ to be the set of with the single element $\{*\}$ if $I \ne
    \emptyset$ and $\emptyset$ if $I = \emptyset$. This is the same generating
    function $F_S = e^x - 1$ (note the lack of constant term).

    If I have an arbitrary graph with $n$ vertices and I draw edges between any
    symmetric subset of the pairs of vertices (edges), The first thing you get
    is a breakdown of the graph into connected components. If I want to count
    the connected graphs, relate it to generating functions of graphs that I do
    know.
\end{exmp}

\section{February 13, 2017}

\subsection{Trees}

\begin{defn}[Tree]
    A tree is a graph without cycles (or a minimal connected graph). Key fact:
    there exists only one simple path between any two vertices, so there are
    only $n - 1$ edges in a tree..
\end{defn}

\begin{defn}
    If $f^n$ is constant for some $n>0$ the we say that $f$ is \textit{nilpotent}.
\end{defn}

\begin{exmp}
    How many trees are there with a given vertex set $I$ with $n$ vertices?

    The answer is $n^{n-2}$. We will use the theory of species to count the
    things that we want to count. The cast of characters that will appear in
    this argument are:
    \begin{itemize}
        \item $S_{tree}[I]$ is the set of trees with vertex set $I$.
        \item $S_{1-tree}[I]$ is the set of trees with vertex set $I$ and a
            distinguished vertex $i \in I$ (the root). If $I = \emptyset$ then
            this returns the singleton set.
        \item $S_{2-tree}[I]$ is the set of trees with vertex set $I$ and two
            distinguished vertices (and ordered pair of vertices, both can be
            the same).
        \item $S_{end}[I]$ (endomorphism) is the set of all maps $f: I \rightarrow I$.
        \item $S_{perm}$ is the set of permutations of $I$.
        \item $S_{lin}$ is the set of linear orderings of $I$.
        \item $S_{nil}$ is the set of nilpotent maps $I \rightarrow I$.
        \item $I_0$ is the set of periodic elements for a map $\pi$. If $\pi$
            is a permutation then $I_0 = I$ and if $\pi$ is nilpotent then $I_0
            = \{i_0\}$.
    \end{itemize}

    The claim we are going to prove is that $|S_{2-tree}[<n>]| = n^n =
    |S_{end}[<n>]|$. But note that $S_{2-tree} \ne S_{end}$.
\end{exmp}

\begin{lemma}[$S_{end} = S_{perm} * S_{nil}$]
    We have to establish a bijection between these two species on any given
    set. Suppose we start with an arbitrary map $\pi: I \rightarrow I$.  Say $i
    \in I$ is periodic for $\pi$ if $\pi^n(i) = i$ for some $n > 0$. 

    $\pi: I_0 \rightarrow I_0$ is a permutation of $I_0$. For any $i \in I$,
    the sequence $i, \pi(i), \pi^2(i), \ldots$ eventually has to repeat. If it
    repeats, then it repeats after the first occurence of a periodic element.

    Define a function $r: I \rightarrow I_0$ by saying that $r(i)$ is the first
    periodic element in the afformentioned sequence. If $i \in I_0$ then $r(i)
    = i$. $r$ defines an equivalence relation on $I$ so that $i \equiv j
    \Leftrightarrow r(i) = r(j)$. The equivalence classes $J$ are indexed by
    elements in $I_0$, and an equivalence class consists of all the points $i$
    the gravitate towards a given periodic element. $\pi$ is nilpotent on $J$
    with attractor $r(J)$. So we're grouping the elements $I$ into subsets
    indexed by elements in $I_0$ where the map on the equivalence classes is
    nilpotent.

    $\pi$ gives me 3 things: an equivalence relation on $I$, a permutation of
    the set of equivalence classes, and a nilpotent map $J \rightarrow J$ on
    each equivalence class. But this is exactly what we want to prove: an
    endomorphism gives rise to a permutation and a nilpotent map.
\end{lemma}

\begin{lemma}[$S_{nil} = S_{1-tree}$]
    The map $\pi: I \rightarrow i$ is nilpotent with attractor $I_0$ then we
    describe the tree by declaring that $i$ is adjacent to $\pi(i) \forall i
    \ne i_0$. Conversely, if you have a tree with a root $i_0$, you can define
    a map by mapping a vertex to the next vertex in its simple path to the
    root.
\end{lemma}

\begin{lemma}[$S_{2-tree} \equiv S_{lin} \circ S_{1-tree}$]
    Let $(T, v, v') \in S_{2-tree}[I]$. Let $I_0 = \{v_0, \ldots, v_n\}$ and
    Define map $r: I \rightarrow I_0$. $r(w)$ is the first point where the unique
    simple path from $w$ to $v$ hits $I_0$.

    This gives rise to an equivalence relation on $I$, for each equivalence
    class $J$, a rooted tree with vertex set $J$, and a linear ordering of the
    set of equivalence classes.
\end{lemma}

\begin{lemma}
    $F_{S_{end}} = F_{S_{perm}} \circ F_{s_{nil}} - F_{S_{perm}} \circ
    F_{S_{1-tree}} = F_{S_{lim}} \circ F_{S_{1-tree}} = F_{S_{2-trees}}$.
\end{lemma}

\section{February 15, 2017}

\subsection{Groups}

\begin{defn}[Group action]
    Take a group $G$ and a set $X$. By a (left)-action of $G$ on $X$ we mean a
    map $G \times X \rightarrow X$ for which $(g, x) \rightarrow g(x) = gx$
    such that $ex =x \forall x \in X, g(hx) = (gh)x \forall g, h \in G, x \in
    X$. $\forall g \in G$, the map $\{g\} \times X \rightarrow X$ is a
    bijection, so we have a map $G \rightarrow Perm(X)$. The two conditions
    listed above amount to saying that this map is a group homomorphism.
\end{defn}

\begin{defn}[$G$-set]
    A $G$-set is a set $X$ with an action of $G$.
\end{defn}

\begin{defn}[$G$-set isomorphism]
    An isomorphism of $G$-sets $X, X'$ is a bijection $\phi: X \rightarrow X'$
    such that $\phi(gx) = g(\phi x)$.
\end{defn}

\begin{defn}[Orbit]
    Say $X$ is a $G$-set. For $x \in X$, the orbit $Gx$ of $x$ under $G$ is the
    image of $G \times \{x\} \rightarrow X$. The orbits form a partition of
    $X$, so $X$ is a disjoint union of orbits: $\forall x, y$ either $Gx = Gy$
    or $Gx \cap Gy = \emptyset$. Being in some orbit is an equivalence
    relation. We define $G \setminus X$ to be the set of orbits.

    The proto-question of this unit is going to be to describe the cardinality
    of $G \setminus X$.
\end{defn}

\begin{defn}[Fixed point set]
    $X$ is a $G$-set. $\forall g \in G$, define $X^g = \{x \in X : gx = x\}
    \subset X$ is the fixed point set.
\end{defn}

\begin{defn}[Stabilizer]
    $\forall x \in X$, define $\mathrm{stab}(x) = \{g \in G : gx = x\} \subset
    G$ is a subgroup.
\end{defn}

\begin{defn}[Transitive $G$-set]
    $X$ is a transitive $G$-set if it has a unique orbit.
\end{defn}

\begin{exmp}
    Say $H \subset G$ is any subgroup. $X = G \setminus H$ is the set of left
    cosets $gH$ which isn't a group unless $H$ is normal. $G$ acts on $X$ in an
    obvious way: $g'(gH) = (g'g)H$. Observe that this is transitive: to get
    from $gH$ to $g'H$ then multiply on the left by $g'g^{-1}$, so you can get
    from any coset to any other coset.
\end{exmp}

\begin{prop}
    Every transitive action of $G$ is of the form $G \setminus H$ for some $H$.
    i.e. if $X$ is any $G$-set, we have an isomorphism $X\equiv \coprod G
    \setminus H_i$ for some collection $\{H_i\}$ of subgroups of $G$.
\end{prop}

\begin{proof}
    The identity coset of $G \setminus H$ is $H$ and the stabilizer of $id(G
    \setminus H)$ is the subgroup $H$ itself. Check that we have an isomorphism
    $x \cong G \setminus H$.
\end{proof}

\begin{exmp}
    Fix $t > 0$. How many ways are there of coloring the faces of a tetrahedron
    with $t$ colors. $X$ is the set of 4 faces of the tetrahedron, and $G = A_4$ is
    the set of 12 rotational symmetries of the tetrahedron. The action of $G$
    on $X$ is of the form $\coprod G \setminus H_i$.
\end{exmp}

\begin{exmp}[Burnside formula]
    Let $X$ be a $G$-set. Then $|G \setminus X| = \frac{1}{|G|} \sum_{g \in G}
    |X^g|$. So the trivial action is $G \times X \rightarrow X$ defined by $(g,
    x) \rightarrow x$, and the formula reads $|G \setminus X| = |X|$ and $|X^g|
    = |X|$ so the formula becomes trivial.
\end{exmp}

\begin{defn}[Free action]
    An action of $G$ on $X$ is free if the action carries no element to itself:
    $\forall g \ne e, X^g = \emptyset$. In other words, $gx = x \Rightarrow g =
    e$ for any $x$. What are the orbits? The map $G \times \{x\} \rightarrow X$
    is an inclusion (the size of the orbit $Gx$ has size $|G|$). And $|G
    \setminus X| = |X|/|G|$.
\end{defn}

\section{February 17, 2017}

Chapter 2 of Artin's Algebra book is on the course website in the files
section.

\subsection{Groups}

It is useful to think of groups as symmetries of an object. A square has four
90-degree symmetries, a non-square rectangle has two 180-degree symmetries. But
it's not jsut the number of symmetries that concerns us - it's not just the
cardinality of the number of symmetries, but the fact that the symmetries
satisfies the group axioms. For many applications, you need to know the group
structure in addition to the cardinality of things.

\begin{exmp}
    $G$ is a group acting on a set $X$. If $H \subset G$ is a subgroup, then we
    can form the set of cosets of $H$ $gH \forall g \in G$. $G$ acts on $X$ by
    $g': gH \rightarrow g'gH$.
\end{exmp}

\begin{prop}
    Every transitive $G$-set is of the form $G/H$. Any $G$-set is of the form
    $\coprod G/H$. 
\end{prop}

\begin{remark}[Burnside formula]
    $X$ is a $G$-set, then $|G/X| = \frac{1}{|G|} \sum_{g \in G} |X^g|$ where
    $X^g = \{ x \in X: gx = x\}$.
\end{remark}

\begin{proof}
    We can assume $X = G/H$ and then we have to show in this case that $|G| =
    \sum_{g \in G} |X^g|$. So $X^g = \{g'H \in G/H : gg'H = g'H \} = \{g'H :
        g'^{-1}gg' \in H \} = \{g'H : g^{'-1}gg' \in H\}$.

    In particular, $|X^g|$ is going to be the number of cosets such that
    $g^{'-1}gg' \in H\}$. Every coset of $H$ has the same cardinality, so
    instead of counting cosets we can count elements of $G$ and then divide by
    teh cardinality of $H$. So, $|X^g| = |\{g' \in G : g^{'-1}gg' \in H\}| /
    |H|$. The key step is sum the previous expression over all $g \in G$. Note
    that $|g'Hg^{'-1}| = |H|$.
    \begin{align*}
        \sum_{g \in G} |X^g| &= \frac{1}{|H|} |\{(g, g') \in G \times G : g^{'-1}gg' \in H\}|\\
        &= \frac{1}{|H|} |\{(g, g') : g \in g'Hg^{'-1}\}| \\
        &= \frac{|G|}{|H|}|H|= |G| 
    \end{align*}
\end{proof}

\begin{exmp}
    How many ways are there to color the faces of a tetrahedron if $t$ colors
    are available (we don't have to use them all) up to rotational symmetry?

    Let $F$ be the set of faces ($|F| = 4$). We want to consider $X$, the
    colourings of $F$ with colors in $T$ where $|T| = T$ so that $X = T^F$
    where $T^F$ is the set of maps from faces to colors. The question we ask
    is: if $G$ is the group of rotational symmetries of the tetrahedron, then
    $G$ acts on $F$ by definition and so $G$ acts on the set of maps $T^F: F
    \rightarrow T$. So the question is how many orbits of there on the latter
    map: what is the cardinality of $G/T^F$.

    Claim: there are three types of symmetries. The identity, for which the
    orbits are just the faces and so there are 4; rotation by 120 degrees about
    the center of a face, there are 8 elements like this and each one has two
    orbits; finally, midpoints of the edges with rotations of 180-degrees, of
    which there are three.

    In current situation, $|G| = 12$. For a given $g$, what is $|X^g|$? If $g =
    e$, then every coloring is fixed and so we have a choice of any one of $t$
    colors and so $|X^e| = t^4$. If $g$ is a 120-degree rotation, then $|X^g| =
    t^2$. If $g$ is a 180-degree rotation, then $|X^g| = t^2$.

    Conclusion: $|G \ T^F| = \frac{1}{12} (t^4 + 8t^2 + 3t^2) = \frac{1}{12}
    (t^4 + 11t^2)$. Check that this is always an integer.
\end{exmp}

In general, if $G$ acts on a set $X$, for $g \in G$ we can consider the number
of $g$-orbits = $o(g)$. The Burnside formula implies that $|G \ T^X| =
\frac{1}{|G|} \sum_{g \in G} t^{o(g)}$.

\section{February 22, 2017}

\subsection{Unlabelled Graphs}

How many graphs are there with vertex set $\{1, \ldots, n\}$? How many graphs
with $n$ vertices are there, up to isomorphism (modulo some kind of symmetry)?
For the first question, we look at all possible sets of edges. An edge is a
two-element subset, so there are $2^{n \choose 2}$. For the second question,
the answer is much less clear. Basic tool: Burnside formula: for a finite group
$G$ which acts on a set $X$, then the number of orbits in $X$ $|G/X| =
\frac{1}{|G|} \sum_{g \in G} |X^g|$.

\begin{exmp}[Polya's Enumeration Theorem]
    How many ways are there to color $X$ with a finite set $T$ of $t$ colors?
    This is to say, how many maps $T^X$ (up to symmetries of $G$)?

    For a given $g \in G$, let $o(g)$ be the number of $g$-orbits in $X$. For a
    coloring to be preserved under $g$, you have to choose a color for each
    orbit of $g$ on $X$. Burnside says: $|G/T^X| = \frac{1}{|G|} \sum_{g \in G}
    t^{o(g)}$. This is Polya's enumeration theorem.
\end{exmp}

\begin{exmp}[Number of unlabelled graphs on $n$ vertices]
    Polya's theorem as expressed seems specific to coloring, but it's not. It
    can be used for the unlabelled graph problem. Let $<n> = \{1, \ldots, n\}$
    and associate to this set $E = E_n$ which is the set of unordered pairs of
    distinct elements of $<n>$ (or, equivalently, subsets of $<n>$ with
    cardinality 2.) So $|E| = {n \choose 2}$. A graph on $<n>$ is an arbitrary
    subset of $E$, so the set of graphs with vertex set $<n>$ is exactly the
    set of all subsets of $E$.  Since $\Sigma_n$ acts on $<n>$ it also acts on
    $E$ and so it acts on the set of subsets of $E$ (graphs on $<n>$).

    How do we set this up as a coloring problem? To specify a subset of a given
    set, you can think of it as a coloring of the set by two colors (in and
    out).  Color it 0 if its in the subset and 1 if its not. So the subsets of
    $|E|$ are going to be exactly the colorings of $E$ with two elements. 
    
    We're trying to find $|\Sigma_n / \{0, 1\}^E| = \frac{1}{n!} \sum_{g \in
    \Sigma_n} 2^{o(g)}$ where $o(g)$ is the number of orbits of $g$ acting on
    $E$.
\end{exmp}

\begin{exmp}[How many graphs on $4$ vertices?]
    In all these situations, you just have to consider the conjugacy classes of
    the symmetric group. There are 5 types (conj. classes) of permutations
    $\sigma \in \Sigma_4$. $|E| = {4 \choose 2} = 6$.
    \begin{enumerate}
        \item $e$ - there's 1, there are 6 orbits each with 1 element.
        \item $(ab)$ - there are 6, 4 orbits (2 fixed points, 2 orbits of size 2).
        \item $(abc)$ - there are 8, 2 orbits (each of size 3).
        \item $(abcd)$ - there are 6, 2 orbits (sizes 2 and 4).
        \item $(ab)(cd)$ - there are 3, 4 orbits (2 fixed opints, 2 orbits of size 2).
    \end{enumerate}
    So the answer is $\frac{1}{24} (2^6 + 6(2^4) + 8(2^2) + 6(2^2) + 3(2^4)) =
    11$.

    Let's organized them by number of edges: there's 1 graph with no edges,
    there's 1 graph with 1 edge, etc. You'll notice there's a symmetry between
    a graph and the complementary graph by taking the edges on vertices that
    aren't connected in the first graph. 

    We haven't talked about the number of conjugacy classes of $\Sigma_n$, but
    that grows fairly fast. There isn't a closed-form formula for this.
\end{exmp}

The next question is: how many graphs are there with $n$ vertices and $k$
edges? We'll call this number $c_k$.

Let $F(t) = \sum_{k = 0}^{n \choose 2} c_kt^k$. We know that $F(1) = \sum c_k =
\frac{1}{n!} \sum_{\sigma \in \Sigma_n} 2^{o(\sigma)}$. Let $E$ be the set of
unordered pairs of elements of $<n>$. Let $X_k$ be the set of graphs with
vertex set $<n>$ with $k$ edges which is the same thing as the set of subsets
of $E$ having cardinality $k$. So $|X_k| = {{n \choose 2} \choose k}$.

By Burnside formula, $c_k = |\Sigma_n \ X_k| = \frac{1}{n!} \sum_{\sigma \in
\Sigma_n} |X^\sigma_k|$. For each $\sigma \in \Sigma_n$, look at action of
$\sigma$ on $E$. Say this has orbits $E_1, E_@, \ldots, E_{o(\sigma)}$. We want
a collection of orbits whose sizes add up to $k$. $|X_k^\sigma|$ is the set of
subsets of $E$ of size $k$ consisting of a union of orbits. How many of these
are there? One way to keep track is to consider the following polynomial: $ (1
+ t^{|E_1|})(1 + t^{E_2|})\ldots(1 + t^{|E_{o(\sigma)}|})$. What is the
coefficient of $t^k$ in this product? This corresponds exactly to collections
of orbits whose total cardinality adds up to $k$.

\section{February 22, 2017}

\subsection{Wreath Products}

Basic construction: say for composite $n$ that $n = kl$ . Take set $S = <n>$
 and express $S = S_1 \coprod S_2 \ldots \coprod S_k$ where $|S_i| = l$.
Consuder the group $K$ of permutations of $S$ that preserve this decomposition
into subsets. We choose an equivalence relation with $k$ equivalence classes,
each of cardinality $l$. We want to consider the group of permutations that
preserve the equivalence relation. 

So we have a map $K \rightarrow \Sigma_k$. The kernel of this map comprises
permutations that contain any of the $S_i$ to itself. So $\cong (\Sigma_l)^k$;
in particular, $|K| = k!(l!)^k$. As a set: $K \cong \Sigma_k \times
(\Sigma_l^k)$ (not as a group). $K$ is called the wreath product of $\Sigma_l$
and $\Sigma_k$.

Now say $G \subset \Sigma_k$. Equivalently say $X, Y$ are sets of size $k, l$
and $G$ acts on $X$ and $H$ acts on $Y$.

Now we consider permutations of $X \times Y \rightarrow X$.

\section{March 20, 2017}

\subsection{Inclusion/Exclusion Principle}

This week: inclusion/exclusion principle and partially ordered sets. In the
notes, this starts with lecture 18. We're now three lectures behind the notes.

Three questions:
\begin{enumerate}
    \item
        How many derangements of $< n >$ are there?
    \item
        How many integers between $1$ and $n$ inclusive are relatively prime to
        $n$?
    \item
        How many surjective maps $f: <m> \rightarrow <n>$ are there?
\end{enumerate}

In each, if you don't specify a condition then the answer is pretty immediate
(e.g. remove "relatively prime" or "surjective"). Then we just exclude the
non-surjective or non-relatively prime elements.

\subsection{Cardinality of set unions}

Starting point: if $X, Y$ are any sets, what is the $|X \cup Y|$? If they are
disjoint, then $|X \cup Y| = |X| + |Y|$; otherwise, it is strictly less than
this sum, so $|X \cup Y| = |X| + |Y| - |X \cap Y|$. If $X, Y, Z$ any sets, then
$|X \cup Y \cup Z| = |X| + |Y| + |Z| - |X \cap Y| - |X \cap Z| - |Y \cap Z| +
|X \cap Y \cap Z|$. You have to add in the last term because you are removing
elements in all three sets three times and adding them three times without it.
General formula: if $X_1, \ldots, X_n$ are any sets, what is $|\cup X_i|$? Let
$X_J = \cap_{i \in J} X_i$. For $x \in X$, let $I_x = \{i : x \in X_i$. Note
    that the empty set is included in the second formulation (using the
    complement equivalence). 

\begin{align*}
|\cup X_i| &= \sum_{J \subset \{1, \ldots, n\}, J \ne \emptyset} (-1)^{|J| + 1} |X_j| \\
&= \sum_{J \subset \{1, \ldots n\}} (-1)^{|J|} |X_J| \\
&= \sum_J \sum_{x \in X} (-1)^{|J|} \text{ if $x \in X_J$ and 0 otherwise} \\
&= \sum_{x \in X} \left( \sum_{J \subset \{1, \ldots, n\}}  (-1)^{|J|} \text{ if $x \in X_J$ and 0 otherwise} \right) \\
&= \sum_{x \in X} \left(\sum_{J \subset I_x} (-1)^{|J|} \right) \\
&= |\{x \in X: I_x = \emptyset\}| = |X - \cup X_i|
\end{align*}

For a two element set, two subsets have odd elements and two have even
cardinality. So the expression sums to 0 except with the emptyset. For
$\emptyset$, then $(-1)^{|J|} = 1$, and so $|\cup X_i| = 1$.

\subsection{Derangements}

Let $X = \mathrm{Perm}(<n>)$ and $X_i = \{ \sigma \in I: \sigma(i) = i\}$.
Then, $\mathrm{Derangements}(<n>) = X - \cup X_i$. If $|J| = k$, then $|X_J| =
(n - k)!$.  The conclusion is that the number of derangments is $|X - \cup
X_i|$:

\begin{align*}
    |X - \cup X_i| &= \sum_{J \subset \{1, \ldots, n\}} X_J \\
    & X_J \{ \sigma \in X: \sigma(i) = i \forall I \subset J\} \\
    &= \mathrm{Perm}(<n> - J) \\
    & |X_J| = \sum_{k = 0}^n \sum_{|J| = k} (-1)^k (n - k)!  \\
    &= \sum_{k = 0}^n (-1)^k {n \choose k}(n - k)! \\
    &= \sum_{k \ge 0} (-1)^k \frac{n!}{k!} \\
    &= n! \sum (-1)^k / k!  \\
    &= n!(1 - \frac{1}{1!} + \frac{1}{2!} \ldots) \\
    &\sim \frac{n!}{e}
\end{align*}

\subsection{Surjective maps}

How many surjective maps $f: <m> \rightarrow <n>$. Let $X$ be the set of all
maps so that $|X| = n^m$. The way to test whether a map is surjective is to see
if there is an element in the target set not in the image of the function.  Let
$X_i = \{f : i \in Im(f)\}$ for $i = 1, \ldots, n$. So $|X_i| = (n - 1)^m$
because you're just excluding $i$ from the target set and then collecting all
other maps.  So $|X_J| = (n - |J|)^m$. 

So the number of surjective maps is:
\begin{align*}
    |X - \cup X_i| &= \sum_{J \in <n>} (-1)^{|J|} (n - |J|)^m \\
    &= \sum_{k = 0}^n (-1)^k { n \choose k } (n - k)^m
\end{align*}

\subsection{Euler $\phi$-function}

The question is how many integers between 1 and $n$ are relatively primte to
$n$? The answer is denoted as $\phi(n)$ and is classically denoted as the Euler
$\phi$-function.

Let $X = \{1, \ldots, n\}$. Let $S$ be the set of primes $p$ such that $p | n$.
$\forall p \in S$, let $x_p \{i \in X : p | i\}$. Let $J = \{,p_1, p_2, \ldots
p_k\} \subset S$. So, 
\begin{align*}
    |X_J| &= \{i \in X : p_1p_2\ldots p_k | x\} \\
    |X_J| &= \frac{n}{p_1p_2\ldots p_k} \\
    \frac{\phi(n)}{n} &= \sum_{J \subset S} (-1)^{|J|} \frac{1}{p_1p_2\ldots p_k} \\
    &= \sum_{J \subset S} \prod_{p \in J} \frac{-1}{p} \\
    &=  \sum_{J \subset S} \prod_{p \in S} \left(\frac{-1}{p} \text{ if $p \in J$ and 1 if $p \notin J$} \right) \\
    &= \prod_{p \in S}\left(1 - \frac{1}{p}\right) \\
\end{align*}

So if $n = p_1^{a_1}p_2^{a_2}\ldots$ written in prime factor form, then
$\phi(n) = (p_1 - 1)p_1^{a_1 - 1}(p_2 - 1)p_2^{a_2 - 1} \ldots$. 

\subsection{Partially Ordered Sets}

\begin{defn}
    A partially ordered set is a set $A$ with a binary relation $\le$ with the following properties:
    \begin{itemize}
        \item Reflexivity: $a \le a \forall a \in A$
        \item
            Transitivity: $a \le b \centerdot b \le c \Rightarrow a \le c
            \forall a, b, c, \in A$
        \item Anti-symmetry: $a \le b$ and $b \le a$ then $b = a$.
    \end{itemize}
    You can think of a partially ordered set as a subset of $A \times A$ that
    contains the diagonal and is anti-symmetric (harder to express transitivity). 
\end{defn}

We're going to hold off on assuming that $A$ is finite at this point. But we
are still going to think of $A$ as basically a finite set, but we are now in
Axiom of Choice territory so watch out. Joe has told us that when you counter a
new notion you ought to look at the extreme cases. So how big can this subset
of $A \times A$ be? 

\begin{exmp}
    Let $A = P(S)$ where $A$ is the powerset of $S$. We can say that $T \le U
    \Leftrightarrow T \subset U$. Another case is where $A = \mathbb{Z}$ and $x
    \le y$ if $x | y$. The extreme examples: define $a \le b$ iff $a = b$ (the
    \textbf{discrete/trivial ordering}). The opposite extreme is where either
    $a \le b$ or $a \le b$ for any pair $(a, b)$. This is called a
    \textbf{total/linear ordering}.  So if $A \subset \mathbb{R}$ and you
    define $a \le b$ if $b - a \ge 0$.
\end{exmp}

The reason to look at these extreme examples is because if I give you an
arbitrary partially ordered set, to what extent can you find subsets of the
extremes?

\end{document}
